%!TEX TS-program = xelatex
\documentclass[11pt]{article}

% חבילות נדרשות
\usepackage{fontspec}
\usepackage{polyglossia}
\usepackage{graphicx}
\usepackage{xcolor}
\usepackage{geometry}
\usepackage{tikz}
\usepackage{float}
\usepackage{enumitem}
\usepackage{hyperref}
\usepackage{fancyhdr}
\usepackage{listings}
\usepackage{tcolorbox}

% הגדרות שפה ופונטים
\setmainlanguage{hebrew}
\setotherlanguage{english}
\newfontfamily\hebrewfont[Script=Hebrew]{David CLM}
\newfontfamily\hebrewfonttt[Script=Hebrew]{Miriam Mono CLM}
\newfontfamily\englishfont{Times New Roman}

% הגדרות עימוד
\geometry{margin=2.5cm}
\parskip=0.5em
\linespread{1.15}

% הגדרת צבעים
\definecolor{schoolgreen}{RGB}{47, 79, 79}
\definecolor{notegray}{RGB}{240, 240, 240}

% הגדרת תיבת הערה למילוי
\newtcolorbox{fillbox}{
  colback=notegray,
  colframe=schoolgreen,
  boxrule=0.5pt,
  left=5pt,
  right=5pt,
  fontupper=\small
}

% הגדרת כותרות עליונות ותחתונות
\pagestyle{fancy}
\fancyhf{}
\fancyhead[R]{\texthebrew{נדיר האקרמן }}
\fancyhead[L]{\texthebrew{חיזוי תוצאות בחירות 2025}}
\fancyfoot[C]{\thepage}

\begin{document}

% עמוד שער
\begin{titlepage}
\begin{center}

% לוגו בית הספר
\includegraphics[width=0.4\textwidth]{school-logo}

\vspace{2cm}

{\Huge\textbf{פרויקט גמר}}

\vspace{1cm}

{\LARGE\textbf{בנושא:}}

\vspace{0.5cm}

{\LARGE\textbf{\textcolor{schoolgreen}{חיזוי תוצאות בחירות 2025}}}

\vspace{2cm}

{\Large
מוגש במסגרת לימודי הנדסת תוכנה\\
התמחות למידת מכונה ולמידה עמוקה\\
5 יחידות לימוד
}

\vspace{2cm}

\begin{Large}
\begin{tabular}{r@{}l}
\textbf{מגיש/ה:} & \hspace{2cm} נדיר האקרמן\\[0.3cm]
\textbf{ת"ז:} & \hspace{2cm} 123456789\\[0.3cm]
\textbf{מנחה:} & \hspace{2cm} שי פרח
\end{tabular}
\end{Large}

\vfill

{\Large \textbf{מאי \LR{2025}}}

\end{center}
\end{titlepage}

% תוכן עניינים
\tableofcontents
\newpage

% רשימת איורים ותרשימים
\listoffigures
\newpage

\section{מבוא ותקציר מנהלים}
\begin{fillbox}
כאן יש לכלול:
\begin{itemize}
  \item קישור לסרטון הדגמה של כ-2 דקות
  \item תיאור קצר של הבעיה והפתרון המוצע
  \item הסבר על החשיבות והרלוונטיות
  \item תיאור תמציתי של הפתרון והתוצאות
  \item סקירה קצרה של פתרונות דומים
\end{itemize}
\end{fillbox}

\section{רקע תיאורטי}
\begin{fillbox}
יש לכלול:
\begin{itemize}
  \item עקרונות הלמידה העמוקה
  \item מבנה ועקרונות רשתות נוירונים
  \item אלגוריתמי אופטימיזציה ופונקציות הפעלה
  \item תוספת לפי סוג הפרויקט:
  \begin{itemize}
    \item[CV:] עקרונות ראייה ממוחשבת ועיבוד תמונה
    \item[RL:] עקרונות למידת חיזוק ו-DQN
    \item[GAN:] ארכיטקטורת Generator-Discriminator
  \end{itemize}
\end{itemize}
\end{fillbox}

\section{מערך הנתונים/סביבת העבודה}
\begin{fillbox}
יש למלא את החלק הרלוונטי לסוג הפרויקט:

\textbf{עבור Computer Vision:}
\begin{itemize}
  \item תיאור ה-Dataset ומקורו
  \item התפלגות הנתונים וניתוח סטטיסטי
  \item תהליכי עיבוד מקדים
  \item אוגמנטציות
  \item פיצול הנתונים
\end{itemize}

\textbf{עבור Reinforcement Learning:}
\begin{itemize}
  \item תיאור המשחק וחוקיו
  \item מרחב המצבים והפעולות
  \item פונקציית התגמול
  \item מימוש הסביבה
\end{itemize}

\textbf{עבור GAN:}
\begin{itemize}
  \item תיאור נתוני האימון
  \item תהליכי הכנה ועיבוד מקדים
  \item שיטות נרמול
  \item מדדי איכות
\end{itemize}
\end{fillbox}

\section{ארכיטקטורת הרשת}
\begin{fillbox}
יש לכלול:
\begin{itemize}
  \item תרשים כללי של הרשת
  \item פירוט השכבות והקישורים
  \item הסבר על בחירת הארכיטקטורה
  \item תוספות ספציפיות לתחום (CV/RL/GAN)
\end{itemize}

מומלץ להוסיף:
\begin{itemize}
  \item תרשים מפורט של כל שכבות הרשת
  \item טבלת פרמטרים
  \item הסבר על כל החלטה ארכיטקטונית משמעותית
\end{itemize}
\end{fillbox}

\section{תהליך האימון}
\begin{fillbox}
\textbf{מרכיבים משותפים:}
\begin{itemize}
  \item היפר-פרמטרים
  \item תהליך האופטימיזציה
  \item גרפי התכנסות
  \item אתגרים ופתרונות
\end{itemize}

\textbf{תוספות לפי תחום:}\\
יש להוסיף את החלק הרלוונטי לפרויקט שלכם
\end{fillbox}

\section{תוצאות והדגמות}
\begin{fillbox}
יש להציג:
\begin{itemize}
  \item תוצאות כמותיות
  \item דוגמאות מייצגות
  \item ניתוח מקרי הצלחה וכישלון
  \item השוואה למודלים/פתרונות אחרים
\end{itemize}
\end{fillbox}

\section{דיון ומסקנות}
\begin{fillbox}
יש לכלול:
\begin{itemize}
  \item ניתוח התוצאות
  \item השוואה למצב הקיים בתחום
  \item מגבלות ואתגרים
  \item הצעות לשיפור והרחבה
\end{itemize}
\end{fillbox}

\section{רפלקציה אישית}
\begin{fillbox}
תארו את:
\begin{itemize}
  \item תהליך הלמידה והפיתוח
  \item אתגרים והתמודדות
  \item תובנות ולקחים
  \item מה הייתם עושים אחרת
\end{itemize}
\end{fillbox}

\appendix
\section{נספחים}
\begin{fillbox}
ניתן לכלול:
\begin{itemize}
  \item קוד מרכזי עם הסברים
  \item גרפים ותרשימים נוספים
  \item פירוט טכני משלים
  \item תיעוד API (אם רלוונטי)
\end{itemize}
\end{fillbox}

\section{ביבליוגרפיה}
\begin{fillbox}
יש לכלול:
\begin{itemize}
  \item מקורות מידע ומאמרים
  \item תיעוד טכני
  \item מקורות קוד (אם נעשה שימוש בקוד קיים)
\end{itemize}
פורמט APA
\end{fillbox}

\end{document}